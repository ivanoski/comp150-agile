% Please do not change the document class
\documentclass{scrartcl}

% Please do not change these packages
\usepackage[hidelinks]{hyperref}
\usepackage[none]{hyphenat}
\usepackage{setspace}
\doublespace

% You may add additional packages here
\usepackage{amsmath}

% Please include a clear, concise, and descriptive title
\title{Comparing the effective time wasted with and without pair programming in an agile working enviroment}

% Please do not change the subtitle
\subtitle{COMP150 - Agile Development Practice}

% Please put your student number in the author field
\author{1700068}

\begin{document}

\maketitle

\abstract{dgrds}

\section{Introduction}

Pair programming is the system of software development where two programmers work together, one working on the keyboard, the "designated driver" while the other watches, watching out for bugs, potential problems and discussing these with the partner. Pair programming has been shown to provide a more positive experience for both members and has additional benefits such as helping newer programmers learn at a much quicker pace. One of the theoretical problems with pair-programming is that it is less time-efficient in comparison to two solo programmers, who could in theory do twice the amount in the same time, having “two do the work of one”. This essay will assess the validity of that argument and try to judge the pros and cons of pair programming in relation to time management.


\section{Your section title here}

The expectation is that the time spent by two programmers sharing a keyboard is 200percent of what two individual programmers could achieve, however studies show the average was only around 15percent. This can be explained by my studies as i have learnt that the presence of the partner "causes procrastinating to decrease to a minimum", phonecalls are cut short and a focused enviroment is created. This means that the time spent while pair programming is very effectively used in comparison to a single programmer who inevitably becomes more easilly distracted by emails and browsing. "They admit to working “harder and smarter” because they don’t want to let their partner down". 
Even though pair programming wastes less time than the initial assumption of 200 percent, 15 percent longer is still a disadvantage. However this disadvantage is heavilly outweighed by the benefits gained through pair programming. Studies showing that the the programs had 15 percent less defects in the code. This can be explained by the constant peer reviewing of the code that occurs as a side-effect of pair programming. The programmer who is not on the keyboard has the effect of not only pointing out errors in the code but also ways to shorten the code. A third study showing that code which has been pair-programmed has "the same functionality as the individuals in fewer lines of code."

less bugs - more time effecient long term
less lines of code - less lines to read to understand, saving time

\section{Conclusion}

 . a conMcDowell et al. (2002) concluded that “students who programmed in pairs produced better programs, completed the course at higher rates, and performed about as well on the final exam as students who programmed independently."

\bibliographystyle{ieeetran}
\bibliography{references}

\end{document}
