% Please do not change the document class
\documentclass{scrartcl}

% Please do not change these packages
\usepackage[hidelinks]{hyperref}
\usepackage[none]{hyphenat}
\usepackage{setspace}
\doublespace

% You may add additional packages here
\usepackage{amsmath}

% Please include a clear, concise, and descriptive title
\title{Comparing the effective time wasted with and without pair programming in an agile working enviroment}

% Please do not change the subtitle
\subtitle{COMP150 - Agile Development Practice}

% Please put your student number in the author field
\author{1700068}

\begin{document}

\maketitle

\section{Introduction}

Pair programming is the system of software development where two programmers work together, one working on the keyboard, the "designated driver" while the other watches, watching out for bugs, potential problems and discussing these with the partner. Pair programming has been shown to provide a more positive experience for both members and has additional benefits such as helping newer programmers learn at a much quicker pace. One of the theoretical problems with pair-programming is that it is less time-efficient in comparison to two solo programmers, who could in theory do twice the amount in the same time, having “two do the work of one”\cite{. This essay will attempt to analyse the differences between the two systems and understand which one cuts down on the amount of time working on a project, or a series of projects. This essay could help companies (The games industry in particular) evaluate whether to allow, favour or even enforce pair programming to cut down time spent on large pieces of software. The "highest priority" stated by the agile manifesto is  "to satisfy the customer through early and continuous delivery of valuable software."\cite{Agile} If pair-programming can be used to achieve these goals then it should be key part of the agile working enviroment. 

\section{Short term time comparison}

The expectation is that the time spent by two programmers sharing a keyboard is two hundred percent of what two individual programmers would spend working on the same project. Reasons for this assumption being that only one of the programmers is actively making code.  Sources show the average extra time spent through pair programming in order to complete one piece of work is around fifteen percent\cite{Cockburn00thecosts}. This unexpectedly small difference can be explained by to two reasons. The first, an avoidance of being impolite to the coworker. The presence of the partner "causes procrastinating to decrease to a minimum"\cite{Keith:2010:AGD:1830460}, and as studies show "Procrastination approximately consumes more than one fourth of most people’s working days"\cite{Procrastination} .  The second reason is that there is a much slimmer chance of getting stuck on a problem. Problems in the code cause the programmer to slow down and work out bugs and grammatical errors within the code, however in pair programming the "non - driver" offsets this by acting as a constant reviewer of the code, finding problems as the code is written, therefore leading to less time wasted. In summary pair programming is much more time efficient in comparison to a single programmer because of the fact of the lack of distractions and reduction of time spent on bugs, however two individual programmers still outpace a pair programming team by fifteen percent. 


\section{Long term time comparison}

This disadvantage is heavilly outweighed by the benefits gained through pair programming. Studies showing that the the programs had fifteen percent less defects in the code\cite{Williams00strengtheningthe}. This can be explained by the constant peer reviewing of the code that occurs as a side-effect of pair programming. This means that the initial code from the pair programmers will be less likely to need code review, and will be less likely to cause problems down the line due to inital bugs that were overlooked. This saves time in the long run as problems tend to be harder to fix later on in the process especially if caused by bugs created early in the project.
The programmer who is not on the keyboard has the effect of not only pointing out errors in the code but also ways to shorten the code. A third study showing that code which has been pair-programmed has "the same functionality as the individuals in fewer lines of code."\cite{Cockburn00thecosts}. This means that it will take less time for other programmers to read and understand the code when working on previous bugs or new features, therefore reducing development time.


\section{Conclusion}

 . a conMcDowell et al. (2002) concluded that “students who programmed in pairs produced better programs, completed the course at higher rates, and performed about as well on the final exam as students who programmed independently."

\bibliographystyle{ieeetran}
\bibliography{references}

\end{document}
